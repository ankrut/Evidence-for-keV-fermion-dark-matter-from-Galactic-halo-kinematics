\begin{abstract}
The enclosed mass of the Milky Way (MW) at halo scales ($\gtrsim \SI{10}{\kilo\parsec}$) has been recently estimated with good precision through the Galactic globular clusters kinematics data of the second Gaia data release (DR2) and the Hubble Space Telescope (HST). Further estimates are inferred from the kinematics of tidal streams. Especially, the anaylsis from the Sagittarius stream in the outer halo indicates a 'skinny Milky Way'. This implies a Keplerian behavior with a nearly constant enclosed mass when compared with results from the globular cluster kinematics, e.g. $v(r) \propto r^{-1/2}$ for $r \gtrsim \SI{40}{\kilo\parsec}$. We show here that these latest measurements favor clearly the Ruffini-Argüelles-Rueda (RAR) model of fermionic dark matter (DM), for a particle mass of $mc^2 = \SI{48}{\kilo\electronvolt}$, over the Navarro-Frenk-White (NFW) and Burkert phenomenological DM profiles.
\end{abstract}