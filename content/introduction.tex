\section{Introduction}
\label{sec:introduction}
% MW composition
The decomposition of the Milky Way galaxy is a topic of great interest. In its simplest form the Galaxy is composed of a Galactic core (or nucleus), a stellar bulge structure (e.g. inner and main bulge), a stellar disk and a dark matter halo. Following \citet{2013PASJ...65..118S}, then a standard bulge and disk are assumed, e.g. an exponential sphere model for the inner as well as main bulge and an exponential disk model for the disk. Of greater debate is the nature of the Galactic core and halo. The standard assumption is that a BH -- though an inactive one --- harbors in the center of the Galaxy and the halo is described by a phenomenological DM model (e.g. NFW).

% galactic center
Information about the Galactic center (at sub-parsec scales) is provided through the orbital measurements of the S-cluster stars \citep{2009ApJ...707L.114G}. Those stars are very well described by Keplerian orbits, indicating a compact object in the galactic center with a core mass $M_{\rm core} = \SI{4.2E6}{\Msun}$ enclosed within the S2 star pericentre $r_{p(S2)}=6\times 10^{-4} {\rm pc}$.

% Sofue (2013)
The bulk of the baryonic matter dominated region ($r \approx \SI{1}{\parsec}$ -- $\SI{10}{\kilo\parsec}$) is very well covered by the rotation curve of \citet{2013PASJ...65..118S}. The same data provides also circular velocities far beyond $\sim\SI{10}{\kilo\parsec}$, though with large uncertainties what leaves a wide window for different DM models.

% Watkins+ (Gaia + HST)
More recently, kinematics of Galactic globular clusters, measured by Gaia (DR2) and the Hubble Space Telescope, allowed to constraint the enclosed total mass $M(r)$ within a radius $r$ \citep{2018arXiv180411348W}. Thus, such an analysis yields that the enclosed mass of the Milky Way is $0.22^{+0.04}_{-0.03} \times 10^{12}\,M_\odot$ at $\SI{21.1}{\kilo\parsec}$ and $0.40^{+0.08}_{-0.05} \times 10^{12}\,M_\odot$ at $\SI{39.5}{\kilo\parsec}$.

% early estimates at 50kpc
Those results are very good in agreement with early estimates of the enclosed mass at $\SI{50}{\kilo\parsec}$, e.g. $0.54^{+0.02}_{-0.36} \times 10^{12}\,M_\odot$ \citep{1999MNRAS.310..645W} and $0.42 \pm 0.04 \times 10^{12}\,M_\odot$ \citep{2012MNRAS.424L..44D}.

% tidal streams approach (Gibbons+, Küpper+)
The analysis of tidal streams such as the Palomar 5 stream \citep{2015ApJ...803...80K} and the Sagittarius stream \citep{2014MNRAS.445.3788G} give further constraints for the Galactic halo. The former is in good agreement with the results from globular cluster kinematics, giving $0.21 \pm 0.04 \times 10^{12}\,M_\odot$ at $\SI{19}{\kilo\parsec}$. The latter, on the other hand, gives a tight constraint of $0.41 \pm 0.04 \times 10^{12}\,M_\odot$ at the outer halo at $\SI{100}{\kilo\parsec}$.

% Keplerian
It is important to emphasize here, that the series of measurements gives evidence for a nearly constant total MW mass beyond $\sim\SI{40}{\kilo\parsec}$. This implies a tight Keplerian behavior in the rotation curve, e.g. $v(r) \propto r^{-1/2}$.

% aim
We show in this paper that this Keplerian trend can be explained by fermionic dark matter including cutoff effects \citep{2016arXiv160607040A} in combination with a reasonable standard disk \citep{2013PASJ...65..118S}. In contrast, phenomenological DM models, like NFW and Burkert as considered here, cannot explain the Keplerian trend and require a negligible disk component.

% outline
The outline of this paper is as following. In \cref{sec:method} we explain the chosen set of constraints, recall briefly few theoretical models of galactic components, define three different MW composition scenarios and describe our method in fitting the MW. The results of this analysis then are presented in \cref{sec:results}.