\section{Results}
\label{sec:results}
% observables description
The best-fit solutions for each scenario are presented through rotation curves and compared with observational data, see \cref{fig:vrot}. In all plots the selected data set of Sofue's measurements (within $\SI{15}{\kilo\parsec}$) are shown as black dots with error bars. The ignored observables, due to more recent and precise data  (beyond $\SI{15}{\kilo\parsec}$), are shown as light grey dots for comparison. The more recent observables are shown as coloured diamonds with much better precision.

\loadfigure{figures/vrot}

% component description
The observables then are compared with the total rotation curve (baryons + DM) of the best-fit solutions, plotted as a solid line. For a better understanding of the different MW components the baryonic contribution (bulge + disk) is shown as a dash-dotted line. Of more interest is the decomposition of the disk (dotted line) and the DM halo (dashed line).

% NFW
The main result of this analysis is that the scenario with the fermionic DM provides here the best fit and explanation for the Galactic disk and halo. Thus, the Keplerian trend beyond $\sim\SI{40}{\kilo\parsec}$ requires necessary cutoff effects (e.g. evaporation) in the DM halo. Further, the fermionic DM scenario yields a reasonable disk in mass and size.

% NFW and Burkert
In contrast, the NFW and Burkert scenarios cannot explain the Galactic halo, especially the Keplerian trend in the outer halo due the $\rho(r) \propto r^{-3}$ trend in the density profile. This is resembled in the worse least-square values, $\chi^2 = 17.3$ for NFW and $\chi^2 = 15.7$ for Burkert (compared with $\chi^2 = 12.6$ for the RAR scenario). Moreover, both scenarios yield a negligible disk due to the relatively wide maxima of the DM halo. See \cref{tbl:sol:param} for a detailed list of the best-fit parameters for each scenario.

\begin{table*}[tbp]
\centering
\begin{tabularx}{\hsize}{@{\extracolsep{\fill}}llcrrr}
\hline
component   & \multicolumn{1}{c}{parameter} & unit                 & \multicolumn{1}{c}{RAR} & \multicolumn{1}{c}{NFW} & \multicolumn{1}{c}{Burkert} \\
\hline
\hline
core        									& mass $M_{\rm core}$           & $M_\odot$            & $\SI{4.2E6}{}$           & $\SI{4.2E6}{}$           & $\SI{4.2E6}{}$                \\
\hline
\multirow{ 2}{*}{inner bulge} & length scale $R_b$            & pc                   & 5.4                    	& 4.5                    	 & 4.70                        \\
															& density scale $\rho_b$        & $M_\odot/{\rm pc}^3$ & $\SI{1.5E4}{}$           & $\SI{1.8E4}{}$           & $\SI{1.81E4}{}$               \\
\hline
\multirow{ 2}{*}{main bulge}  &    length scale $R_b$         & pc                   & $\SI{1.4E2}{}$           & $\SI{9.8E1}{}$           & $\SI{1.09E2}{}$               \\
															& density scale $\rho_b$        & $M_\odot/{\rm pc}^3$ & $\SI{1.5E2}{}$           & $\SI{1.2E2}{}$           & $\SI{2.05E2}{}$               \\
\hline
\multirow{ 2}{*}{disk} 				& length scale $R_d$            & pc                   & $\SI{2.6E3}{}$           & $\SI{2.3E2}{}$           & $\SI{5.25E2}{}$               \\
															& total mass $M_d$            	& $M_\odot$            & $\SI{4.8E10}{}$          & $\SI{4.1E9}{}$           & $\SI{4.86E9}{}$               \\
\hline
\multirow{ 2}{*}{DM}  				& halo radius $r_h$             & pc                   & $\SI{3.1E4}{}$           & $\SI{1.1E4}{}$           & $\SI{1.1E4}{}$               \\
															& halo mass $M_h$               & $M_\odot$            & $\SI{2.8E11}{}$          & $\SI{9.8E10}{}$          & $\SI{1.1E11}{}$               \\
\hline
\hline
total       									& least-square $\chi^2$         &                      & 12.6                     & 17.3                     & 15.7\\
\hline
\end{tabularx}%
\caption{List of best-fit parameters for the RAR, NFW and Burkert scenario. For a better comparison of the DM models, the length and mass scale of each DM halo are provided where $r_h$ is defined at the maximum in the DM halo rotation curve and $M_h \equiv M_{\rm dm}(r_h)$.}
\label{tbl:sol:param}
\end{table*}

% hard cutoff
We expect that the Keplerian trend in the NFW and Burkert scenarios can be achieved only with a truncation (e.g. when DM density falls below a critical density value) of the phenomenological dark matter distribution. This approach would increase the DM halo in mass and size, allowing more reasonable disk parameters.

% RAR superior
However, we want to emphasize that fermionic dark matter including cutoff effects can explain naturally the Keplerian trend. Additionally, the cutoff effects are responsible for a narrow DM maximum bump in the rotation curve what gives excellent conditions for an accompanying disk with convenient parameters. Moreover, we want to recall that such RAR solutions (e.g. for $mc^2 = \SI{48}{\kilo\electronvolt}$) develop a compact quantum core in the Galactic center as an alternative to the BH scenario in SgrA* \citep{2016arXiv160607040A}.

% summary
In sum, the RAR scenario explains the Keplerian trend, a reasonable disk and the compact quantum core without spoiling the intermediate baryonic matter. This scenario doesn't require the introduction of a BH nor a truncation of the DM density profile. Instead, the Galactic core and halo are naturally explained by fermionic dark matter including cutoff effects. This result gives further evidence for fermionic dark matter in the keV regime.


%\loadfigure{figures/vrotRAR}
%\loadfigure{figures/vrotNFW}
%\loadfigure{figures/vrotBurkert}


%\loadfigure{figures/IsoContourDisk}
%\loadfigure{figures/NFWContourDisk}
%\loadfigure{figures/BurkertContourDisk}
